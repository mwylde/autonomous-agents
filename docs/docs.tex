\documentclass{article}

\usepackage{hyperref}
\usepackage{parskip}
\usepackage{amsthm}
\usepackage{amsmath}
\usepackage{amssymb}
\usepackage{wrapfig}
\usepackage{graphicx}
\usepackage{listings}
\usepackage{color}
\usepackage{fullpage}

\author{Micah Wylde\\Jeffrey Ruberg}
\date{\today}
\title{Autonomous Navigation: The Quest for a Real Title\\Comp 352}

\begin{document}
\maketitle

\section{Introduction}

\subsection{Related work}

\section{Autonomous Driving}

\subsection{Simulation}

\section{Methods}

To create an autonomous agent that navigates while simulating a car's behavior
and traffic laws, we took three general approaches: deliberative planning
through A* search, reactive navigation through a dynamical system, and a hybrid
of deliberative planning and reactive motion. As for the agents' environment, we
create graphical worlds out of real map data. The code architecture consists of
a server and a separate client for each agent.

\subsection{Code Architecture}

The project is written in Ruby, under JRuby to utilize Java2D for the graphical
display. Specifically, all server code runs specifically under JRuby, but client
code can run under any flavor of Ruby. Agents are represented both on the client
and server end; server agents perform motion- and display-related
calculations, and client agents contain all the navigation inference and
decision-making and ultimately send decisions (restricted to behavior variables)
back to the server againt.

A brief description of the various source files will follow.

\begin{description}
\item[app.rb] This file is the point of entry for the program. This same entry
  point is used to, based on various command-line options, start a server, start
  a new agent, or run tests.

\item[client_agent.rb] This file contains the \code{ClientAgent} class, the
  super class which every client agent inherits. Client agents receive messages
  from remote server agents, process the message (based on their form of
  navigation), and then send back a response in their behavior variable space.

\item[constants.rb] This file contains various general constants that may be
  used in several locations or files, or that may be particularly useful to
  tweak with.

\item[display.rb] This file contains all of the display code used to generate
  our rendering of the world.

\item[map.rb] This file contains the classes, specifically \code{Map}, which
  encode information provided from real map data.

\item[pqueue.rb] A priority queue implementation useful for A* search.

\item[remote_agent.rb] This file contains the \code{RemoteServerAgent} class,
  which is a subclass of \code{ServerAgent}. Essentially, a remote server agent
  is a server agent which is tied to a specific client agent and communicates
  with that client agent.

\item[server_agnet.rb] This file contains the \code{ServerAgent} class, which
  contains all the base representation and calculations needed for an agent (for
  example, the server agent computes various points needed to display the agent
  graphically).

\item[server.rb] This file contains the socket server to which new agents
  connect.

\item[socket.rb] MICAH

\item[util.rb]

\item[agents/astar.rb]

\item[agents/dynamical.rb]

\item[agents/hybrid.rb]

\item[agents/simple.rb]

\subsection{Simulation Environment}

\subsection{Deliberative Planning}

\subsection{Reactive Navigation}

\subsection{Hybrid Approach}

\section{Results}

\section{Conclusion}

\end{document}
